\section{Séance 5}

\textbf{Graphes Hamiltoniens}

\paragraph{1. } Pour quelle classe de graphes un tour eulérien est-il aussi hamiltonien? Peut-on en déduire une caractérisation des graphes qui sont à la fois eulériens et hamiltoniens?

\paragraph{2. } Le problème du cavalier était un problème en vogue au XVIIIème siècle. Le problème est de déterminer s'il est possible de faire parcourir toutes les cases d'un échiquier de taille $n \times n$ à un cavalier en ne passant qu'une seule fois par chaque case et en revenant à son point de départ. Formulez cela comme un tour hamiltonien dans un graphe et montrez que la réponse est négative pour $n$ impair.
\begin{solution}
Chaque noeud représente une case de l'échiquier. Les arrêtes sont les déplacements possibles en partant d'une case. L'objectif est alors de trouver une cycle hamiltonien sur ce graphe. Pour le cas où $n$ est impair, on peut tirer parti d'une propriété de ce graphe : on remarque facilement que le graphe est biparti. Or si $n$ est impair, les deux partition des noeud sont de taille différentes. Dés lors il n'est pas possible de trouver un cycle qui parcourt tout les noeuds une et une seule fois et qui revient au point de départ.
\end{solution}

\paragraph{3. } Le théorème de Dirac affirme que si $d(v) > n/2$ pour tous les sommets $v \in V$ d'un graphe $G=(V,E)$, alors $G$ est hamiltonien. Donnez un graphe hamiltonien de $n$ sommets qui ne satisfait pas les conditions du théorème de Dirac. Donnez un graphe qui n'est pas hamiltonien et qui satisfait $d(v) \geq (n-1)/2$ pour tout $v \in V$.

\paragraph{4. } Placez $n$ sommets sur un cercle et liez un sommet à ses deux plus proches voisins dans les deux directions. Démontrez que pour $n \geq 5$ le graphe obtenu est l'union de deux circuits hamiltoniens.

\paragraph{5. } Une souris mange un gruyère de dimension $3 \times 3 \times 3$ par petits morceaux de taille $1 \times 1 \times 1$. Elle démarre en un coin du cube et se déplace de cube adjacent en cube adjacent. Peut-elle manger la totalité du gruyère en terminant par le cube central?

\begin{solution}
Chaque noeud représente une part du cube (27 parts au totale). Les arrêtes sont verticales ou horizontales. Ce faisant on a construit un graphe biparti. Comme il y a 27 noeuds, la bi-partition comporte respectivement 14 et 13 noeuds. Il n'est alors pas possible de construire un cycle qui passe par tous les noeuds et qui revient au point de départ. Une autre manière de voir les choses est d'utiliser la condition nécessaire sur les graphe hamiltoniens: si on supprime les 13 noeuds, on obtient 14 composantes connexes. Or si un graphe est hamiltonien, supprimer $k$ noeuds crée au plus $k$ composantes connexes. On a une contradiction et donc le graphe n'est pas hamiltonien.
\end{solution}

\paragraph{6. } Trouvez des bornes supérieures et inférieures sur la valeur optimale du problème du voyageur de commerce avec les distances suivantes (les distances satisfont l'inégalité triangulaire) :
\begin{equation}
  \left( \begin{matrix}
      - & 5 & 4 & 3 & 6 \\
      - & - & 6 & 4 & 5 \\
      - & - & - & 3 & 3 \\
      - & - & - & - & 5 \\
      - & - & - & - & -
  \end{matrix}  \right)
\end{equation}

\paragraph{7. } Formulez le problème suivant comme le problème du voyageur de commerce. Cinq opérations $j=1,…,5$ doivent avoir lieu sur une machine spécialisée. Les temps de transition $c_{ij}$ entre deux opérations sont importants, et sont donnés dans la table suivante:
\begin{equation}
  \left( \begin{matrix}
      10 & 23 & 42 & 11 & 24 \\
      19 & 13 & 42 & 36 & 43 \\
      67 & 34 & 23 & 29 & 21 \\
      19 & 52 & 41 & 37 & 31 \\
      96 & 63 & 75 & 89 & 43
  \end{matrix}  \right)
\end{equation}
En plus, si une opération $j$ est en première position, il faut aussi un temps de préparation $t_j$ qui est fonction de l'opération, où $t=(23,41,32,54,11)$. Pour des raisons techniques, on décide que l'opération $3$ ne peut pas précéder directement l'opération $1$, et les opérations $4$ et $5$ ne peuvent pas avoir lieu en dernière position.

\paragraph{8. } Un groupe de huit personnes se retrouve pour diner. Le graphe ci-dessous précise les "incompatibilités d'humeur" entre les personnes de ce groupe (une arête reliant deux personnes indique que celles-ci s'entendent très modérément). Proposez un plan de table (la table est ronde) pour ce groupe en évitant de placer côte à côte deux personnes "incompatibles".

\begin{figure}[h!]
  \begin{center}
    \begin{tikzpicture}[-,>=stealth',shorten >=1pt,auto]
      \Vertex[x=0 ,y=0]{8}
      \Vertex[x=0 ,y=-2]{7}
      \Vertex[x=2,y=1]{1}
      \Vertex[x=2 ,y=-3]{6}
      \Vertex[x=4 ,y=1]{2}
      \Vertex[x=4 ,y=-3]{5}
      \Vertex[x=6 ,y=0]{3}
      \Vertex[x=6 ,y=-2]{4}


      \path[every node/.style={font=\sffamily\small}]
      (1) edge node [left] {} (2)
      edge node [left] {} (4)

      (2) edge node [right] {} (5)
      edge node [right] {} (6)
      edge node [right] {} (8)

      (3) edge node [right] {} (4)
      edge node [left] {} (5)

      (4) edge node [right] {} (7)

      (5) edge node [right] {} (6)

      (6) edge node [right] {} (8)

      (7) edge node [left] {} (8);

    \end{tikzpicture}
  \end{center}
\end{figure}

\newpage

\textbf{Couplages}

\paragraph{9. } Etant donné le couplage $M = \left\lbrace  (1,1'), (2,4'),(4,2'),(5,5')  \right\rbrace$ dans le graphe biparti ci-dessous, démontrez que $M$ est maximum.

\begin{figure}[h!]
  \begin{center}
    \begin{tikzpicture}[-,>=stealth',shorten >=1pt,auto]
      \Vertex[x=0 ,y=0]{1}
      \Vertex[x=0 ,y=-1]{2}
      \Vertex[x=0,y=-2]{3}
      \Vertex[x=0 ,y=-3]{4}
      \Vertex[x=0 ,y=-4]{5}
      \Vertex[x=3 ,y=-0]{1'}
      \Vertex[x=3 ,y=-1]{2'}
      \Vertex[x=3 ,y=-2]{3'}
      \Vertex[x=3 ,y=-3]{4'}
      \Vertex[x=3 ,y=-4]{5'}


      \path[every node/.style={font=\sffamily\small}]
      (1) edge [style=dashed]  node [left] {} (1')
      edge node [left] {} (5')

      (2) edge node [right] {} (1')
      edge node [right] {} (2')
      edge node [right] {} (3')
      edge [style=dashed]  node [right] {} (4')
      edge node [right] {} (5')

      (3) edge node [right] {} (1')
      edge node [left] {} (5')

      (4) edge [style=dashed]  node [right] {} (2')
      edge node [right] {} (3')
      edge node [right] {} (4')
      edge node [right] {} (5')

      (5) edge node [right] {} (1')
      edge [style=dashed]  node [left] {} (5');


    \end{tikzpicture}
  \end{center}
\end{figure}


\paragraph{10. } Vrai ou faux? \\
\begin{itemize}
  \item Un arbre possède un couplage parfait si et seulement si tous les chemin d'une feuille à une autre sont de longueur impaire.
  \item Un arbre possède au plus un couplage parfait.
  \item Dans un arbre, si pour tout noeud $u$, il existe une feuille $v$ telle que $d(u,v)$ est impaire, alors cet arbre possède un couplage parfait.
  \item Soit $o(H)$, le nombre de composantes impaires du graphe $H$, c'est-à-dire le nombre de composantes connexes ayant un nombre impair de sommets. Un arbre $G$ admet un couplage parfait si $o(G-v)=1,  \ \ \forall v \in V$.
  \item Si un arbre $G$ admet un couplage parfait, alors $o(G-v)=1, \ \ \forall v \in V$.
\end{itemize}



\paragraph{11. } Thibault a la responsabilité de répartir les séances d'exercices des cours de mathématiques appliquées entre les assistants du pôle INMA. Chacun des assistants donne pour cela à Thibault une liste de ses cours préférés, qui sont repris dans la table ci-dessous.

\begin{center}
  \begin{tabular}{|c|c|}
    \hline
    Assistant & Cours préférés \\
    \hline
    Pierre & Projet, Théorie des Matrices \\
    Romain & Graphes, Modélisation Stochastique \\
    Arnaud & Théorie des Matrices \\
    Adeline & Graphes, Optimisation, Analyse Numérique \\
    Benoit & Théorie des Matrices, Projet \\
    Nicolas & Graphes, Optimisation, Analyse Numérique, \\
            & Modélisation Stochastique, Théorie des Matrices  \\
    \hline
  \end{tabular}
\end{center}

Thibault aimerait bien assigner exactement un cours à chaque assistant en respectant autant que possible leurs préférences. Formulez cela comme un problème de couplage maximum dans un graphe.

Pour l'aider dans sa tâche, Thibault dispose de la répartition de l'année dernière:

\begin{center}
  \begin{tabular}{|c|c|}
    \hline
    Romain & Modélisation Stochastique \\
    Adeline & Optimisation \\
    Nicolas & Théorie des Matrices \\
    Pierre & Projet \\
    \hline
  \end{tabular}
\end{center}

En partant de la répartition de l'année dernière, utilisez l'algorithme hongrois pour aider Thibault à trouver un couplage maximum. Ce couplage est-il parfait? Proposez un argument pour prouver que le couplage trouvé est effectivement maximum.




%Dans le graphe graphe biparti ci-dessous, un couplage possible est $M = \left\lbrace  (1,5'), (2,2'),(3,6'),(5,4')  \right\rbrace$. En partant de ce couplage, utilisez l'algorithme hongrois pour trouver un couplage maximum. Proposez un argument pour prouver que le couplage trouvé est effectivement maximum.

%\begin{figure}[h!]
%\begin{center}
%\begin{tikzpicture}[-,>=stealth',shorten >=1pt,auto]
%   \Vertex[x=0 ,y=0]{1}
%   \Vertex[x=0 ,y=-1]{2}
%   \Vertex[x=0,y=-2]{3}
%   \Vertex[x=0 ,y=-3]{4}
%   \Vertex[x=0 ,y=-4]{5}
%   \Vertex[x=0, y=-5]{6}
%   \Vertex[x=3 ,y=-0]{1'}
%   \Vertex[x=3 ,y=-1]{2'}
%   \Vertex[x=3 ,y=-2]{3'}
%   \Vertex[x=3 ,y=-3]{4'}
%   \Vertex[x=3 ,y=-4]{5'}
%   \Vertex[x=3, y=-5]{6'}
%
%     \path[every node/.style={font=\sffamily\small}]
%     (1) edge  node [left] {} (1')
%        	edge [style=dashed]  node [left] {} (5')
%
%    (2) edge node [right] {} (1')
%    	edge [style=dashed]  node [right] {} (2')
%    	edge node [right] {} (3')
%
%    (3) edge node [right] {} (1')
%     	edge  node [right] {} (2')
%    	edge node [right] {} (3')
%         edge  node [left] {} (5')
%         edge [style=dashed]  node [left] {} (6')
%
%    (4) edge node [right] {} (6')
%
%     (5) edge [style=dashed] node [right] {} (4')
%         edge node [left] {} (6')
%
%     (6) edge node [right] {} (4')
%         edge  node [left] {} (6')
%
%
%\end{tikzpicture}
%\end{center}
%\end{figure}


\paragraph{12. } Deux personnes jouent à un jeu sur un graphe $G$ de la manière suivante: \\

\begin{itemize}
  \item Chacune des deux personnes sélectionne chacune à son tour un sommet $v_1, v_2, v_3, …$ tel que $\forall i > 1, \ \ \ v_i$ est adjacent à $v_{i-1}$.
  \item Un sommet déjà sélectionné ne peut plus être choisi.
  \item La dernière personne à sélectionner un sommet gagne le jeu.
\end{itemize}

Montrer que le premier joueur admet une stratégie gagnante si et seulement si le graphe $G$ n'admet pas de couplage parfait.
