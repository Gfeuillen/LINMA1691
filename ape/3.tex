% Packages utilisés :
%
% \usepackage[french]{babel}
% \usepackage[utf8x]{inputenc}
% \usepackage{amsmath}
% \usepackage{amssymb}
% \usepackage{tikz}
% \usepackage{tkz-graph}
%
% Il y a aussi besoin d'un environnement solution

\section{Séance 3}

\paragraph{1. } Lorsque toutes les arêtes d'un graphe sont de longueur 1, la recherche en largeur dans le graphe permet de trouver le plus court chemin du noeud $s$ au noeud $t$. Quelle est la complexité de cet algorithme? Quel est le pire cas?

\begin{solution}
	Étant donné que cet algorithme doit, au plus, parcourir toutes les arêtes pour trouver le chemin, on a une complexité en $O(\vert E \vert)$, avec $\vert E \vert$ le nombre d'arêtes dans le graphe.\\

	Le pire des cas (celui où le nombre d'arêtes est le plus élevé) est le cas du graphe complet. En effet, par le théorème des poignées de mains, on a que $\vert E \vert = \frac{n(n-1)}{2}$. La complexité est donc $O(n^2)$, avec $n$ le nombre de nœuds dans le graphe.
\end{solution}

\paragraph{2. } Lorsque certaines longueurs dans un digraphe sont négatives il peut être nécessaire de devoir faire la distinction entre la plus courte chaîne et le plus court chemin. Donnez l'exemple d'un graphe pour lequel cette distinction est nécessaire. Cette distinction est-elle nécessaire si tous les circuits sont de longueur positive?

\paragraph{3. } L'algorithme de Dijkstra ne fonctionne correctement que lorsque la longueur des arêtes est positive. Donnez un exemple d'un digraphe dont tous les cycles sont de longueur positive et pour lequel l'algorithme de Dijkstra ne fournit pas la bonne solution. (A faire chez vous) Proposez une modification de l'algorithme de Dijkstra pour qu'il donne la bonne solution, même dans le cas où certaines arêtes sont négatives, si tous les cycles sont de longueur positive.

\paragraph{4. } Soit $d_k(j)$ la longueur du plus court chemin de $s$ à $j$ avec au plus $k$ arêtes. Trouvez une récurrence pour $d_k(j)$. Démontrez que $d_{n-1}(j)$ est la plus courte distance de $s$ à $j$.

\paragraph{5. } Le graphe dirigé ci-dessous est acyclique. Utilisez l'algorithme de Dijkstra pour trouver le plus court chemin du noeud $1$ à chaque autre noeud du graphe.

\begin{center}
  \begin{tikzpicture}[->,>=stealth',shorten >=1pt,auto]
    \Vertex[x=0 ,y=0]{1}
    \Vertex[x=0 ,y=-2]{4}
    \Vertex[x=2,y=1]{2}
    \Vertex[x=2 ,y=-1]{3}
    \Vertex[x=2 ,y=-3]{7}
    \Vertex[x=4 ,y=0]{5}
    \Vertex[x=4 ,y=-2]{6}

    \path[every node/.style={font=\sffamily\small}]
    (1) edge node [left] {12} (2)
    edge node [left] {4} (4)

    (2) edge node [right] {5} (5)

    (3) edge node [right] {4} (2)
    edge node [left] {7} (5)
    edge node [right] {3} (7)

    (4) edge node [left] {3} (3)
    edge node [right] {4} (7)

    (5) edge node [right] {3} (6)

    (7) edge node [left] {9} (6);

  \end{tikzpicture}
\end{center}



\paragraph{7. }
	Vous possédez un billet de $p$ euros et vous souhaitez le changer en pièces de $a_1, a_2, ..., a_k$ euros (tous les montants étant entiers). Est-ce possible? Si oui, avec quel nombre de pièces? Formulez ce problème comme un problème de plus court chemin dans un graphe.


\begin{solution}
	Créons un digraphe de la manière suivante : $p+1$ nœuds numéroté de $0$ à $p$ et des arêtes telles qu'une arête associée à une pièce de valeur $a_k$ relie un nœud $i$ à un nœud $i+a_k$\footnote{à condition de ces deux nœuds à relier soient bien compris entre $0$ et $p$}. Il suffit ensuite d'appliquer l'algorithme de Dijkstra au graphe créé pour chercher le plus petit chemin entre les nœuds $0$ et $p$. Si l'algorithme ne trouve pas de chemin, il est alors impossible d'échanger le montant $p$ avec les pièces disponibles. Vous pouvez regarder quelques exemples ci-dessous.\\

	Premièrement, un exemple où il n'existe pas de parcours du nœud $0$ au nœud $p$ avec $p=3$ et $a_1 = 2$. En effet, il n'est pas possible d'échanger $3$ euros avec des pièces de $2$ euros.\\

\begin{center}
	\begin{tikzpicture}[scale=0.75,transform shape]
		\Vertex[x=-10,y=0]{0}
 		\Vertex[x=-8,y=0]{1}
		\Vertex[x=-6,y=0]{2}
		\Vertex[x=-4,y=0]{3}
		\tikzset{LabelStyle/.style =   {draw}}
	 	\tikzstyle{EdgeStyle}=[bend left]
	 	\Edge(0)(2)
	 	\tikzstyle{EdgeStyle}=[bend right]
		\Edge(1)(3)
 	\end{tikzpicture}
\end{center}

	Deuxièmement, un exemple ou il existe un parcours de $0$ à $p$ avec $p=5$, $a_1=2$ et $a_2=1$. Le chemin le plus court est tracé en rouge et correspond à $2$ pièces de $2$ et une pièce de $1$. Il existe deux autres chemins de même longueur (donc équivalents).\\

\begin{center}
	\begin{tikzpicture}[scale=0.75,transform shape]
		\Vertex[x=-10,y=0]{0}
		\Vertex[x=-8,y=0]{1}
		\Vertex[x=-6,y=0]{2}
		\Vertex[x=-4,y=0]{3}
		\Vertex[x=-2,y=0]{4}
		\Vertex[x=0,y=0]{5}
		\tikzset{LabelStyle/.style =   {draw}}
		\tikzstyle{EdgeStyle}=[bend left,double=red]
		\Edge(0)(1)
		\tikzstyle{EdgeStyle}=[bend left]
		\Edge(1)(2)
		\Edge(2)(3)
		\Edge(3)(4)
		\Edge(4)(5)
		\tikzstyle{EdgeStyle}=[bend right, double = red]
		\Edge(1)(3)
		\Edge(3)(5)
		\tikzstyle{EdgeStyle}=[bend right]
		\Edge(0)(2)
	  	\Edge(2)(4)
 	\end{tikzpicture}
\end{center}

\end{solution}
