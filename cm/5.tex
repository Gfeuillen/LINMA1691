\section{Mariages, couplages et couvertures}
\subsection{Couplage}
\index{couplage}
\begin{mydef}
  Un \emph{couplage} dans un graphe est un ensemble $M$ d’arêtes tel que $M$ ne contient pas de boucles et deux arêtes de $M$ n’ont jamais d’extrêmité en commun.
\end{mydef}

\index{couplage!couplage maximum}
\begin{mydef}
  Un \emph{couplage maximum} est un couplage dont le nombre d’arêtes est maximal.
\end{mydef}

\index{couplage!couplage parfait}
\begin{mydef}
  Un \emph{couplage parfait} est un couplage qui est incident à tous les noeuds.
\end{mydef}

\begin{myrem}
  Un couplage parfait, s’il existe, est maximum.
\end{myrem}

\index{chemin!chemin M-alterné}
\begin{mydef}
  Pour un couplage $M$, un \emph{chemin M-alterné} est un chemin qui passe alternativement par une arête de $M$ et par une arête hors de $M$.
\end{mydef}

\index{chemin!chemin M-augmenté}
\begin{mydef}
  Un \emph{chemin M-augmenté} est un chemin M-alterné dont les noeuds d’origine et de destination ne sont pas incident à une arête de $M$.
\end{mydef}

\begin{mytheo} [Berge]
  Un couplage $M$ est maximum si et seulement s’il n’y a pas de chemin M-augmenté.
  \begin{proof}
     Preuve:
     $ \Longrightarrow $ Soit le couplage M représenté en bleu sur la figure ci-dessous. \addTODO{} (mettre graphe)
     On construit le couplage $ M' = M \Delta P$ où $\Delta$ indique une différence symétrique entre M et P ($ M \Delta P = ( M \backslash P) \cup ( P \backslash M)$ ). Ce nouveau couplage est représenté en mauve.
     $ |M'| = |M| + 1 $. On voit bien que M n'est pas maximal.


     $\Longleftarrow$ Soit M' le couplage maximum représenté ci-dessous \addTODO{} (mettre graphe)
      tel que $ |M'| > |M| $ . Regardons $ M \Delta M'$. On observe que les noeuds dans $ M \Delta M'$ dont de degrés $0,1$ ou $2$. Les noeuds de degrés $2$ ont une arête dans M et une arête dans M'. $\Rightarrow$ $ M \Delta M'$ est une union de noeuds isolés, chemins et cycles. Or, dans un cycle de longueur paire, il y a autant d'arêtes dans M que dans M'. Comme  $ |M'| > |M| $, il faut qu'il existe un chemin de longueur impaire qui commence et termine par une arête de M'. On voit facilement que ce chemin est un chemin M-augmenté.

  \end{proof}
\end{mytheo}
\begin{myexem}
  Exemple \addTODO
\end{myexem}

\begin{mytheo} [Théorème du mariage ou de Hall]
  Un graphe biparti avec bipartition $(X , Y)$ possède un couplage incident à tous les noeuds de $X$ si et seulement si pour tout ensemble $S \subseteq X$ , le nombre de voisins de $S$ est au moins $|S|$.
  \begin{proof}
     Preuve $\Longrightarrow$ Un couplage M incident à tout X crée une fonction injective: $X\mapsto Y$ donc $\forall S \subseteq X: Voisin (S) \geq M(S) $. On a donc que $ |Voisin (S)| \geq |M(S)| = |S| $.
     $\downarrow$ On veut montrer qu'il n'existe pas de couplage incident à tout X et donc qu'il existe un $ S \subseteq X: |voisin (S)| < |S|$. \\
     Soit $M*$ le couplage maximal et $ u \in X$ non incident à $M*$. Prenons les chemins $M*$ alternés partant de u: Soit z l'ensemble des noeuds ainsi rencontrés. $ S = X\cap Z$, $T= Y \cap Z$.
     \textit{Preuve.} On remarque \begin{itemize}
     \item Voisins (S)= T ( par constriuction)
     \item	$M*$ est incident à tout T (sinon on aurait un chemin M-augmenté car un chemin alterné qui part de u et qui arrive dans T peut toujours poursuivre par une arête de $M*$)
     \item $M*$ est incident à tout $S\backslash\lbrace u\rbrace$ par construction
     \item $M*$ crée une bijection entre $S\backslash\lbrace u\rbrace$ et T ( car $M* (S\backslash\lbrace u\rbrace) \subseteq T$ et $M*(T) \subseteq S\backslash\lbrace u\rbrace$.
     \end{itemize}
     $ \Rightarrow  Voisin (S)= T \Longleftrightarrow  |Voisin(s)| = |s|-1 < |s|$

  \end{proof}
\end{mytheo}
\begin{myexem}
  Exemple \addTODO
\end{myexem}

\begin{myrem}
  Un graphe est k-régulier si tous les noeuds sont de degré $k$.
\end{myrem}

\begin{mycorr}
  Tout graphe biparti k-régulier (pour $k > 0$) possède un couplage parfait.
\end{mycorr}

\subsection{Couverture}
\index{couverture de sommets}
\begin{mydef}
  Une \emph{couverture de sommets} d’un graphe est un ensemble de sommets incident à toutes les arêtes.
\end{mydef}

\index{couverture de sommets!minimum}
\begin{mydef}
  Une \emph{couverture de sommets minimum} d’un graphe est une couverture de sommets avec un nombre minimal de sommets.
\end{mydef}

\begin{myrem}
  Si $K$ est une couverture de sommets et $M$ un couplage, alors $|M| \leq |K|$.
\end{myrem}

\begin{myrem}
  Si $K^*$ est une couverture de sommets minimum et $M$ un couplage maximum, alors $|M^*| \leq |K^*|$.
\end{myrem}

\begin{mylem}
  Si $K$ est une couverture de sommets, $M$ un couplage et que $|M| = |K|$, alors $K$ est minimum et $M$ est maximum.
  \begin{proof}
     Preuve:
  \end{proof}
\end{mylem}

\begin{mytheo} [König]
  Dans un graphe biparti, si $K^*$ est une couverture de sommets minimum et $M^*$ un couplage maximum, alors $|M^*| = |K^*|$.
  \begin{proof}
     Preuve \addTODO
  \end{proof}
\end{mytheo}

\subsection{L'algorithme hongrois}
\index{algorithme!algorithme hongrois}
\begin{myalgo}[Algorithme hongrois]
  Algorithme \addTODO
\end{myalgo}
\begin{myexem}
  Exemple \addTODO
\end{myexem}
