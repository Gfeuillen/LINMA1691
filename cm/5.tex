\section{Mariages, couplages et couvertures}
\subsection{Couplage}
\index{couplage}
\begin{mydef}
  Un \emph{couplage} dans un graphe est un ensemble $M$ d’arêtes tel que $M$ ne contient pas de boucles et deux arêtes de $M$ n’ont jamais d’extrêmité en commun.
\end{mydef}

\index{couplage!couplage maximum}
\begin{mydef}
  Un \emph{couplage maximum} est un couplage dont le nombre d’arêtes est maximal.
\end{mydef}

\index{couplage!couplage parfait}
\begin{mydef}
  Un \emph{couplage parfait} est un couplage qui est incident à tous les noeuds.
\end{mydef}

\begin{myrem}
  Un couplage parfait, s’il existe, est maximum.
\end{myrem}

\index{chemin!chemin M-alterné}
\begin{mydef}
  Pour un couplage $M$, un \emph{chemin M-alterné} est un chemin qui passe alternativement par une arête de $M$ et par une arête hors de $M$.
\end{mydef}

\index{chemin!chemin M-augmenté}
\begin{mydef}
  Un \emph{chemin M-augmenté} est un chemin M-alterné dont les noeuds d’origine et de destination ne sont pas incident à une arête de $M$.
\end{mydef}

\begin{mytheo} [Berge]
  Un couplage $M$ est maximum si et seulement s’il n’y a pas de chemin $M$-augmenté.
  \begin{proof}
     Preuve:
$ \Longrightarrow $ Soit le couplage $M$, représenté en bleu sur la figure ci-dessous, et un chemin $M$-augmenté en vert que nous noterons $P$. 
     
\begin{center}
    \begin{tikzpicture}[scale=1]
      \SetGraphUnit{1}
      \SetVertexNoLabel
      \Vertex{A}
      \EA(A){B}
      \EA(B){C}
      \SO(C){D}
      \WE(D){E}
      \EA(C){F}
      \EA(F){G}
      \SO(G){H}
      \Edges(C,D)
      \SetUpEdge[color=green]
      \Edges(A,B)
      \Edges(C,F)
      \Edges(G,H)
      \tikzset{EdgeStyle/.style={color=blue,thick,double=green,double distance = 1.2pt}}
      \Edges(F,G)
      \Edges(B,C)
      \SetUpEdge[color=blue]
      \Edges(D,E)
    \end{tikzpicture}
\end{center}        
     
     On construit le couplage $ M' = M \Delta P$ où $\Delta$ indique une différence symétrique entre $M$ et $P$ ($ M \Delta P = ( M \backslash P) \cup ( P \backslash M)$ ). Ce nouveau couplage est représenté en rouge.
     $ |M'| = |M| + 1 $. On voit bien que M n'est pas maximal.

\begin{center}
    \begin{tikzpicture}[scale=1]
      \SetGraphUnit{1}
      \SetVertexNoLabel
      \Vertex{A}
      \EA(A){B}
      \EA(B){C}
      \SO(C){D}
      \WE(D){E}
      \EA(C){F}
      \EA(F){G}
      \SO(G){H}
      \Edges(C,D)
      \Edges(F,G)
      \Edges(B,C)
      \SetUpEdge[color=red]
      \Edges(G,H)
      \Edges(C,F)
      \Edges(A,B)
      \Edges(D,E)
    \end{tikzpicture}
\end{center} 

     $\Longleftarrow$ Soit $M'$ le couplage maximum représenté ci-dessous tel que $ |M'| > |M| $. 

\begin{center}
\begin{tikzpicture}
\SetVertexNoLabel
\GraphInit[vstyle=Normal]
\SetGraphUnit{1}
\begin{scope}[rotate=90]
\Vertices{circle}{A,B,C,D}
\end{scope}
\begin{scope}[rotate=90,shift={(0,-3.5)}]
\Vertices{circle}{E,F,G,H}
\end{scope}
\begin{scope}[rotate=90,shift={(0,3.5)}]
\Vertices{circle}{I,J,K,L}
\end{scope}
\EA(H){M}
\Edges(F,G)
%\Edges(I,A)
\Edges(J,L,B,A,C,D,F,H,G,C)
\SetUpEdge[color=red]
\Edges(B,C)
\Edges(F,E)
\Edges(H,M)
\Edges(J,I)
\Edges(K,L)
\SetUpEdge[color=blue]
\Edges(J,K)
\Edges(I,L)
\Edges(E,H)
\tikzset{EdgeStyle/.style={color=blue,thick,double=red,double distance = 1.2pt}}
\Edges(A,D)
\end{tikzpicture}
\end{center}     
     
     Regardons $ M \Delta M'$. 
     
\begin{center}
\begin{tikzpicture}
\SetVertexNoLabel
\GraphInit[vstyle=Normal]
\SetGraphUnit{1}
\begin{scope}[rotate=90]
\Vertices{circle}{A,B,C,D}
\end{scope}
\begin{scope}[rotate=90,shift={(0,-3.5)}]
\Vertices{circle}{E,F,G,H}
\end{scope}
\begin{scope}[rotate=90,shift={(0,3.5)}]
\Vertices{circle}{I,J,K,L}
\end{scope}
\EA(H){M}
\SetUpEdge[color=red]
\Edges(B,C)
\Edges(F,E)
\Edges(H,M)
\Edges(J,I)
\Edges(K,L)
\SetUpEdge[color=blue]
\Edges(J,K)
\Edges(I,L)
\Edges(E,H)
\end{tikzpicture}
\end{center}
     
     On observe que les noeuds dans $ M \Delta M'$ dont de degrés $0,1$ ou $2$. Les noeuds de degrés $2$ ont une arête dans $M$ et une arête dans $M'$. $\Rightarrow$ $ M \Delta M'$ est une union de noeuds isolés, chemins et cycles. Or, dans un cycle de longueur paire, il y a autant d'arêtes dans $M$ que dans $M'$. Comme  $ |M'| > |M| $, il faut qu'il existe un chemin de longueur impaire qui commence et termine par une arête de $M'$. On voit facilement que ce chemin est un chemin $M$-augmenté.

  \end{proof}
\end{mytheo}
\begin{myexem}
  Exemple \addTODO
\end{myexem}

\begin{mytheo} [Théorème du mariage ou de Hall]
  Un graphe biparti avec bipartition $(X , Y)$ possède un couplage incident à tous les noeuds de $X$ si et seulement si pour tout ensemble $S \subseteq X$ , le nombre de voisins de $S$ est au moins $|S|$.
  \begin{proof}
     Preuve $\Longrightarrow$ Un couplage M incident à tout X crée une fonction injective: $X\mapsto Y$ donc $\forall S \subseteq X: Voisin (S) \geq M(S) $. On a donc que $ |Voisin (S)| \geq |M(S)| = |S| $.
     $\downarrow$ On veut montrer qu'il n'existe pas de couplage incident à tout X et donc qu'il existe un $ S \subseteq X: |voisin (S)| < |S|$. \\
     Soit $M*$ le couplage maximal et $ u \in X$ non incident à $M*$. Prenons les chemins $M*$ alternés partant de u: Soit z l'ensemble des noeuds ainsi rencontrés. $ S = X\cap Z$, $T= Y \cap Z$.
     \textit{Preuve.} On remarque \begin{itemize}
     \item Voisins (S)= T ( par constriuction)
     \item	$M*$ est incident à tout T (sinon on aurait un chemin M-augmenté car un chemin alterné qui part de u et qui arrive dans T peut toujours poursuivre par une arête de $M*$)
     \item $M*$ est incident à tout $S\backslash\lbrace u\rbrace$ par construction
     \item $M*$ crée une bijection entre $S\backslash\lbrace u\rbrace$ et T ( car $M* (S\backslash\lbrace u\rbrace) \subseteq T$ et $M*(T) \subseteq S\backslash\lbrace u\rbrace$.
     \end{itemize}
     $ \Rightarrow  Voisin (S)= T \Longleftrightarrow  |Voisin(s)| = |s|-1 < |s|$

  \end{proof}
\end{mytheo}
\begin{myexem}
  Exemple \addTODO
\end{myexem}

\begin{myrem}
  Un graphe est k-régulier si tous les noeuds sont de degré $k$.
\end{myrem}

\begin{mycorr}[du théorème de Hall]
  Tout graphe biparti k-régulier (pour $k > 0$) possède un couplage parfait.
\end{mycorr}

\subsection{Couverture}
\index{couverture de sommets}
\begin{mydef}
  Une \emph{couverture de sommets} d’un graphe est un ensemble de sommets incident à toutes les arêtes.
\end{mydef}

\index{couverture de sommets!minimum}
\begin{mydef}
  Une \emph{couverture de sommets minimum} d’un graphe est une couverture de sommets avec un nombre minimal de sommets.
\end{mydef}

\begin{myrem}
  Si $K$ est une couverture de sommets et $M$ un couplage, alors $|M| \leq |K|$.
\end{myrem}

\begin{myrem}
  Si $K^*$ est une couverture de sommets minimum et $M$ un couplage maximum, alors $|M^*| \leq |K^*|$.
\end{myrem}

\begin{mylem}
  Si $K$ est une couverture de sommets, $M$ un couplage et que $|M| = |K|$, alors $K$ est minimum et $M$ est maximum.
  \begin{proof}
     Preuve: on a, par définition, $|M| \leq |M^*| \leq |k^*| \leq |k|$. par conséquent, si $|M| = |K|$, $|M| = |M^*| = |K^*| = |K|$
  \end{proof}
\end{mylem}

\begin{mytheo} [König]
  Dans un graphe biparti, si $K^*$ est une couverture de sommets minimum et $M^*$ un couplage maximum, alors $|M^*| = |K^*|$.
  \begin{proof}
     Preuve \addTODO
  \end{proof}
\end{mytheo}

\begin{myexem}[Problème du site de rencontres]
	Nous avons 5 filles et 5 garçons (noeuds) et nous devons en accoupler un maximum en fonction de leurs préférences (arêtes = volonté de s'accoupler). Par conséquent, nous devons réaliser un couplage maximum sur un graphe biparti. Voici le schéma obtenu :
	\begin{center}
	  \begin{tikzpicture}
      \node[vertex] at (-2, 2) (a) {A};
			\node[vertex] at (-1, 2) (b) {B};
			\node[vertex, fill = red] at (0, 2) (c) {C};
			\node[vertex, fill = red] at (1, 2) (d) {D};
			\node[vertex] at (2, 2) (e) {E};
			\node[vertex] at (-2, -2) (f) {a};
			\node[vertex, fill = red] at (-1, -2) (g) {b};
			\node[vertex] at (0, -2) (h) {c};
			\node[vertex, fill = red] at (1, -2) (i) {d};
			\node[vertex] at (2, -2) (j) {e};
			
      \draw[red] (a) edge node {} (g);
			\draw[] (a) edge node {} (i);
			\draw[] (b) edge node {} (g);
			\draw[] (c) edge node {} (f);
			\draw[] (c) edge node {} (g);
			\draw[red] (c) edge node {} (h);
			\draw[] (c) edge node {} (i);
			\draw[] (c) edge node {} (j);
			\draw[] (d) edge node {} (h);
			\draw[red] (d) edge node {} (j);
			\draw[red] (e) edge node {} (i);
  	\end{tikzpicture}
  \end{center}
On voit que $|M|=|K|$ et, par conséquent (voir en rouge), le couplage est maximum (et la couverture est minimum).
\end{myexem}

\subsection{L'algorithme hongrois}
\index{algorithme!algorithme hongrois}
\begin{myalgo}[Algorithme hongrois]
  Algorithme \addTODO
\end{myalgo}
\begin{myexem}
  Exemple \addTODO
\end{myexem}
