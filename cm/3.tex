\section{Arbres et connectivité}
\label{sec-3}
\subsection{Arbres}
\index{arbre}
\index{forêt}
\begin{mydef}
  Un \emph{arbre} est un graphe connexe et sans cycle. Une \emph{forêt} est un graphe sans cycle.
\end{mydef}

\index{graphe!sous-graphe sous-tendant}
\index{graphe!sous-graphe couvrant}
\begin{mydef}
  Un \emph{sous-graphe sous-tendant} ou \emph{couvrant} d’un graphe $G$ est un sous-graphe qui contient tous les sommets de $G$.
\end{mydef}

\begin{mytheo} [Arbres sous-tendants]
  Tout graphe connexe contient un arbre sous-tendant.
  \begin{proof}
     Parmi tous les sous-graphes sous-tendants connexes, on choisit un sous-graphe \emph{minimal} dans cet ensemble (\emph{minimum} = si j'enlève n'importe quelle arête, alors on perd la connexité). Je prétends que c'est un arbre sous-tendant : 
     \begin{itemize}
     \item \emph{Connexe} : par construction
     \item \emph{Sans cycle} : suppose qu'on a un cycle. Soit $e = uv$ une arête quelconque de ce cycle. Si on enlève $e$, le graphe est toujours connexe (en effet, $\forall$ parcours $x \rightarrow uev \rightarrow y$, on peut remplacer $e$ par le reste du cycle dont $e$ faisait partie).
     Il y a contradiction, il n'y a donc pas de cycle.
     \end{itemize}
  \end{proof}
\end{mytheo}

\begin{mytheo} [Caractérisations des arbres]
  Soit $G$ un graphe à $n$ sommets et $m$ arêtes. Alors les conditions suivantes sont équivalentes :
  \begin{itemize}
    \item $G$ est connexe et sans cycle;
    \item $G$ est sans cycle et $m = n − 1$;
    \item $G$ est connexe et $m = n − 1$;
    \item $G$ est connexe et supprimer une arête quelconque déconnecte $G$;
    \item $G$ est sans cycle et ajouter une arête quelconque crée un et un seul cycle;
    \item Deux noeuds de $G$ sont toujours reliés par un seul chemin.
  \end{itemize}
  La dernière condition implique que G est sans boucle (pour deux noeuds identiques).
  \begin{proof}
  Nous allons démontrer que chaque condition implique la suivante et qu'elles sont ainsi toutes équivalentes.  
  \begin{itemize}
  \item $(1) \Rightarrow (2)$ :\\
  Sans cycle : OK\\
  $m = n-1$ : 
  \begin{enumerate}[$\bullet$]
  \item D'abord, on prouve que tout arbre possède une feuille (= nœud de degré 1). Tous les nœuds ont un degré $\geq$ 1 (par connexité, s'il y a au moins deux nœuds). Supposons que tous les nœuds aient un degré $\geq$ 2. Alors, partant d'un nœud on peut créer un parcours sans jamais rebrousser chemin (chaque fois qu'on visite un nœud pour la première fois, on peut sortir par une autre arête), jusqu'à revenir à un nœud déjà visité. Cela implique qu'on peut extraire un cycle $\rightarrow$ contradiction. Tout arbre possède donc au moins une feuille.
  \item On procède par récurrence sur $n$. Pour $n=1, m=0$ : on a bien $m = n − 1$. Pour $n+1$ avec une feuille $x$ : on enlève $x$ et son arête, on obtient un nouvel arbre. Il reste $n$ nœud et (par récurrence) $n-1$ arêtes. Donc on avait $n+1$ nœud et $n$ arêtes.
  \end{enumerate}
  
  \item $(2) \Rightarrow (3)$ :\\
  $m = n-1$ : OK\\
  Connexe : supposons qu'il ne soit pas connexe. Alors au moins 2 composantes sont connexes, de $n_i$ nœuds et $m_i$ arêtes. Sur chacune, on peut appliquer $(1) \Rightarrow (2)$ (puisque chaque composante est connexe et sans cycle). Donc, pour toute composante connexe $i$ : $m_i = n_i -1$. Sommons maintenant cela sur l'ensemble des composantes. Somme : $m = \sum_i m_i = \sum_i n_i - \# \text{composantes connexes} = n - \# \text{comp. conn.} = n - 1$. Donc il n'y a qu'une seule composante connexe. 
  
  \item $(3) \Rightarrow (4)$ :\\ 
  Connexe : OK\\
  Supprimer une arête déconnecte le graphe : supposons par l'absurde qu'on puisse enlever une arête $e$ à $G$ et $G-e$ reste connexe. Par le théorème précédent, dans $G-e$ (connexe), il y a un arbre sous-tendant. Par $(1) \Rightarrow (2)$, cet arbre a $n-1$ arêtes. Donc $G-e$ a au moins $n-1$ arêtes. Donc $G$ a au moins $\geq n$ arêtes $\rightarrow$ contradiction (car hypothèse : $m = n-1$).
  
  \item $(4) \Rightarrow (5)$ :\\ 
  Sans cycle : s'il y avait un cycle, on pourrait supprimer une arête de ce cycle et maintenir la connexité, ce qui n'est pas le cas par hypothèse. Il n'y a donc pas de cycle.\\
  Ajouter une arête quelconque crée un et un seul cycle : 
  \begin{enumerate}[$\bullet$]
  \item D'abord, on prouve qu'il existe un cycle. Ajoutons une arête $e$ entre $u$ et $v$. Par connexité, il existe un chemin $c = u \ldots v$. Donc $\underbrace{u \ldots v}_{c}eu$ est un cycle.
  \item Ensuite, on prouve que c'est le seul cycle. Supposons qu'on ait obtenu au moins 2 cycles $c_1, c_2$. Alors on a un parcours fermé $u \ldots v \ldots u$, dont on peut extraire un cycle dans $G$ $\rightarrow$ contradiction.
  \end{enumerate}
  
  \item $(5) \Rightarrow (6)$ :\\
  Soit deux nœuds $u$ et $v$. Supposons qu'il existe deux chemins $P_1 = u \ldots v$, $P_2 = u \ldots v$, alors $u \ldots v \ldots u$ est un parcours fermé, donc on peut extraire un cycle $\rightarrow$ contradiction. Il y a donc au plus un chemin $u \rightarrow v$. En ajoutant une arête $e$ entre $u$ et $v$, par hypothèse je crée un cycle. Cela implique que $C-e$ est un chemin $u \rightarrow v$. Il y a donc bien un chemin entre deux noeuds de $G$ et c'est le seul.
  
  \item $(6) \Rightarrow (1)$ :\\
  Connexe : OK\\
  Sans cycle : supposons qu'il existe un cycle. Soient $x$ et $y$ deux nœuds dans ce cycle. Le cycle donne deux chemins $x \rightarrow y$ c'est qui est en contradiction avec l'hypothèse. Il n'y a donc pas de cycle.
  
  \end{itemize}
  \end{proof}
\end{mytheo}

\begin{myform} [Formule de Cayley]
  Soit $T(G)$ le nombre d’arbres sous-tendants de $G$, et $e$ une arête quelconque de $G$, qui n’est pas une boucle. \\
  Alors $T(G) = T(G − e) + T (G.e)$.
  \begin{proof}
  On divise les arbres sous-tendants de $G$ en deux catégories
  \begin{enumerate}[a)]
  \item Ceux qui contiennent l'arête $e$
  \item Ceux qui ne contiennent pas l'arête $e$
  \end{enumerate}
  On voit que le nombre de $a)$ sont en bijection avec les arbres sous-tendants de $G.e$ (détails laissé au lecteur)\\
  On voit que le nombre de $b)$ sont en bijection avec les arbres sous-tendants de $G-e$ (détails laissé au lecteur)\\
  On a donc $T(G) = T(G − e) + T (G.e)$.
  \end{proof}
\end{myform}

\begin{mytheo} [Théorème de Cayley]
  Le nombre d’arbres sous-tendants de $K_n$ est $n^{n−2}$ .
  \begin{proof}
     Preuve \addTODO
  \end{proof}
\end{mytheo}

\subsection{Algorithme de Kruskal}
\index{algorithme!algorithme de Kruskal}
\begin{myalgo}[Algorithme de Kruskal]
  Algorithme \addTODO
\end{myalgo}

\begin{myexem}
  \href{https://dl.dropboxusercontent.com/u/44092863/Graph_Theory_Romain_Capron.pdf}{Voir notes} \addTODO
\end{myexem}

\begin{mytheo}
  L’algorithme de Kruskal est correct.
  \begin{proof}
     Preuve \addTODO
  \end{proof}
\end{mytheo}

\begin{mytheo} [L’algorithme de Kruskal est efficace]
  L’algorithme de Kruskal requiert un temps de calcul de l’ordre de $m.log(m)$ sur un graphe à $m$ arêtes.
  \begin{proof}
     Preuve \addTODO
  \end{proof}
\end{mytheo}

\index{coupe de sommets}
\begin{mydef}
  Pour un graphe connexe, une \emph{coupe de sommets} est un ensemble de sommets qui déconnecte le graphe quand on l’en retire.
\end{mydef}

\index{coupe de d'arêtes}
\begin{mydef}
  Pour un graphe connexe, une \emph{coupe de d'arêtes} est un ensemble d’arêtes qui déconnecte le graphe quand on l’en retire.
\end{mydef}

\index{graphe!graphe k-connexe}
\begin{mydef}
  Un graphe est dit \emph{k-connexe} si retirer $k − 1$ noeuds quelconques laisse le graphe connexe. Autrement dit, si toutes les coupes de sommets sont de taille au moins $k$.
\end{mydef}

\index{connectivité}
\begin{mydef}
   La \emph{connectivité} d’un graphe est la taille de la plus petite coupe de sommets. Si tous les $n$ noeuds sont voisins (ex., le graphe complet), la connectivité est définie comme $n − 1$.
\end{mydef}

\index{graphe!graphe k-arête-connexe}
\begin{mydef}
   Un graphe est dit \emph{k-arête-connexe} si retirer $k − 1$ arêtes quelconques laisse le graphe connexe. Autrement dit, si toutes les coupes d’arêtes sont de taille au moins $k$.
\end{mydef}

\index{connectivité!arête-connectivité}
\begin{mydef}
   L’\emph{arête-connectivité} d’un graphe est la taille de la plus petite coupe d’arêtes.
\end{mydef}

\begin{mytheo} [Lien entre les connectivités]
 $$ \text{connectivité} \leq \text{arête-connectivité} \leq \text{degré minimum}$$ 
  \begin{proof}
     Preuve \addTODO
  \end{proof}
\end{mytheo}

\begin{mytheo} [Théorème de Whitney]
  Un graphe à au moins trois noeuds est 2-connexe ssi toute paire de noeuds distincts est reliée par au moins deux chemins dont les noeuds internes sont distincts.
  \begin{proof}
     Preuve \addTODO
  \end{proof}
\end{mytheo}

Ce théorème se généralise :

\begin{mytheo} [Théorème de Menger]
  Un graphe à au moins $k + 1$ noeuds est k-connexe ssi toute paire de noeuds distincts est reliée par au moins deux chemins dont les noeuds internes sont distincts.
  \begin{proof}
     Preuve \addTODO
  \end{proof}
\end{mytheo}

\begin{mytheo} [Nombre d'arêtes dans un graphe k-connexe]
  Tout graphe k-connexe à $n$ noeuds possède $kn/2$ arêtes au moins.
  \begin{proof}
     Preuve \addTODO
  \end{proof}
\end{mytheo}

\begin{mytheo} [Théorème de Harary]
  Le graphe de Harary $H_{k ,n}$ possède $kn/2$ arêtes et est k-connexe.
  \begin{proof}
     Preuve \addTODO
  \end{proof}
\end{mytheo}

\begin{myexem}
  Exemples de graphes de Harary. \addTODO
\end{myexem}
